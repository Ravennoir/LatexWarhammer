%%%%%%%%%%%%%%%%%%%%%%%%%%%%%%%%%%%%%%%%%%%%%%%%%%%%%%%%%%%%%%%%%%%%%
%
% LateX Template of Björn Schröder
% v0.1
%
%%%%%%%%%%%%%%%%%%%%%%%%%%%%%%%%%%%%%%%%%%%%%%%%%%%%%%%%%%%%%%%%%%%%%


%%%%%%%%%%%%%%%%%%%%%%%%%%%%%%%%%%%%%%%%%%%%%%%%%%%%%%%%%%%%%%%%%%%%%
%%%%% Standard Praeamble %%%%%%%%%%%%%%%%%%%%%%%%%%%%%%%%%%%%%%%%%%%%
%%%%%%%%%%%%%%%%%%%%%%%%%%%%%%%%%%%%%%%%%%%%%%%%%%%%%%%%%%%%%%%%%%%%%

%\documentclass{tufte-book}
\documentclass[10pt,a4paper,openright,oneside
    %,twocolumn
    ]{memoir}
\pagestyle{empty}
\usepackage[top=0.8in, bottom=1.25in, left=0.75in, right=2.25in, marginparwidth=2in]{geometry}
%\usepackage{showframe}

\usepackage[T1]{fontenc}
\usepackage[utf8]{inputenc}
%\usepackage[ngerman]{babel} % Hyphenation patterns if we wanted German
    \renewcommand{\listtablename}{bla} 
    \renewcommand{\listfigurename}{bla} 
    \renewcommand{\contentsname}{bla}

%%%%%%%%%%%%%%%%%%%%%%%%%%%%%%%%%%%%%%%%%%%%%%%%%%%%%%%%%%%%%%%%%%%%%
%%%% My Standard packages
%%%%%%%%%%%%%%%%%%%%%%%%%%%%%%%%%%%%%%%%%%%%%%%%%%%%%%%%%%%%%%%%%%%%%
\usepackage{booktabs}
\usepackage{graphicx}
\usepackage{tabularx}
\usepackage{microtype} 
\usepackage{xcolor}
\renewcommand\colorchapnum{\color{ared}}
\renewcommand\colorchaptitle{\color{ared}}
    \definecolor{ared}{rgb}{.647,.129,.149}
    \definecolor{titlepagecolor}{cmyk}{1,.60,0,.40}
    \definecolor{namecolor}{cmyk}{1,.50,0,.10} 
    \definecolor{mygreen}{rgb}{0,0.6,0}
    \definecolor{mygray}{gray}{0.95}
    \definecolor{mymauve}{rgb}{0.58,0,0.82}
%%%%%%%%%%%%%%%%%%%%%%%%%%%%%%%%%%%%%%%%%%%%%%%%%%%%%%%%%%%%%%%%%%%%%
%%%% Useful links and references
%%%%%%%%%%%%%%%%%%%%%%%%%%%%%%%%%%%%%%%%%%%%%%%%%%%%%%%%%%%%%%%%%%%%%
% http://tex.stackexchange.com/questions/19264/techniques-and-packages-to-keep-up-with-good-practices
% http://detexify.kirelabs.org/classify.html
% http://tex.stackexchange.com/questions/553/what-packages-do-people-load-by-default-in-latex

%%%%%%%%%%%%%%%%%%%%%%%%%%%%%%%%%%%%%%%%%%%%%%%%%%%%%%%%%%%%%%%%%%%%%
%%%%%% List of useful packages and associated usecases
%%%%%% Comment if you dont need them
%%%%%%%%%%%%%%%%%%%%%%%%%%%%%%%%%%%%%%%%%%%%%%%%%%%%%%%%%%%%%%%%%%%%%


\RequirePackage[l2tabu,orthodox]{nag}% Old habits die hard. All the same, there are
                                     % commands, classes and packages which are 
                                     % outdated and superseded. nag provides routines
                                     % to warn the user about the use of those.
%\usepackage{amssymb}
%\usepackage{yfonts}
%\usepackage{sidecap}
%\usepackage{textcomp}
%\usepackage{mathptmx}
%\usepackage{courier}
%\usepackage{latexsym}
%\usepackage{amsmath}
%\usepackage{amstext}

\usepackage{listings}
\lstset{ %
    backgroundcolor=\color{mygray},  % choose the background color
    basicstyle=\footnotesize,        % the size of the fonts that are used for the code
    breakatwhitespace=false,         % sets if automatic breaks should only happen at 
                                     % whitespace
    breaklines=true,                 % sets automatic line breaking
    captionpos=b,                    % sets the caption-position to bottom
    commentstyle=\color{mygreen},    % comment style
    deletekeywords={...},            % if you want to delete keywords from the given language
    escapeinside={\%*}{*)},          % if you want to add LaTeX within your code
    extendedchars=true,              % lets you use non-ASCII characters; for 8-bits 
                                     % encodings only, does not work with UTF-8
    frame=single,                    % adds a frame around the code
    keepspaces=true,                 % keeps spaces in text, useful for keeping indentation 
                                     % of code (possibly needs columns=flexible)
    language=Python,                 % the language of the code
    morekeywords={*,...},            % if you want to add more keywords to the set
    numbers=left,                    % where to put the line-numbers; possible values are 
                                     % (none, left, right)
    numbersep=5pt,                   % how far the line-numbers are from the code
    numberstyle=\tiny\color{mygray}, % the style that is used for the line-numbers
    rulecolor=\color{black},         % if not set, the frame-color may be changed on 
                                     % line-breaks within not-black text (e.g. comments 
                                     % (green here))
    showspaces=false,                % show spaces everywhere adding particular underscores; 
                                     % it overrides 'showstringspaces'
    showstringspaces=false,          % underline spaces within strings only
    showtabs=false,                  % show tabs within strings adding particular underscores
    stepnumber=2,                    % the step between two line-numbers. If it's 1, each 
                                     % line will be numbered
    stringstyle=\color{mymauve},     % string literal style
    tabsize=2,                       % sets default tabsize to 2 spaces
    title=\lstname                   % show the filename of files included with 
                                     % \lstinputlisting; also try caption instead of title
}

\newcommand{\inlc}[1]{\colorbox{mygray}{\lstinline{#1}} } 

%\usepackage[style=apa,backend=biber]{biblatex} % APA citations
%% I used to use this:
%\usepackage[sort&compress]{natbib}
%    \bibpunct{[}{]}{,}{s}{}{}
%    \bibliographystyle{Bjoern}

\usepackage[some]{background}
\usepackage{color}
\usepackage{blindtext}
%\usepackage{pifont}
%\usepackage{acronym}
%\usepackage[textstyle,squaren]{SIunits}
%    \addunit{\M}{\textsc{m}}
%    \addunit{\CV}{\textsc{cv}}
%    \addunit{\dalton}{\textsc{D}a}
%    \addunit{\mol}{mol}
\usepackage{abstract}
\usepackage{ifthen}
\usepackage{url}
\usepackage{fancybox}
\usepackage{array}
\usepackage{longtable}
\usepackage{floatflt}
\usepackage{rotating}
%\usepackage{lastpage}
%\usepackage{currvita}
%\usepackage{python}
\usepackage[small,nooneline,bf]{caption}

\usepackage{todonotes}
    %\todo{Rewrite this answer \ldots}
    %\listoftodos
\usepackage{hyperref}
    \hypersetup{
        pdftoolbar=true,        % show Acrobat’s toolbar?
        pdfmenubar=true,        % show Acrobat’s menu?
        pdffitwindow=false,     % window fit to page when opened
        pdfstartview={FitH},    % fits the width of the page to the window
        pdftitle={title},       % title
        pdfauthor={Björn Schröder},     % author
        pdfsubject={Subject},   % subject of the document
        pdfcreator={Björn Schröder},   % creator of the document
        pdfproducer={Björn Schröder}, % producer of the document
        pdfnewwindow=true,      % links in new window
        colorlinks=true,       % false: boxed links; true: colored links
        % color of internal links (change box color with linkbordercolor)
        linkcolor=ared,          
        citecolor=ared,        % color of links to bibliography
        filecolor=ared,      % color of file links
        urlcolor=black           % color of external links
        }
\usepackage[all]{hypcap} % fixes jump behaviour of hyperref and must be loded afterwards


%%%%%%%%%%%%%%%%%%%%%%%%%%%%%%%%%%%%%%%%%%%%%%%%%%%%%%%%%%%%%%%%%%%%%
% Coverpage
%%%%%%%%%%%%%%%%%%%%%%%%%%%%%%%%%%%%%%%%%%%%%%%%%%%%%%%%%%%%%%%%%%%%%

%\backgroundsetup{
%    scale=1,
%    angle=0,
%    opacity=1,
%    contents={\begin{tikzpicture}[remember picture,overlay]
%     \path [fill=titlepagecolor] (current page.west)rectangle (current page.north east); 
%     \draw [color=white, very thick] (5,0)--(5,0.5\paperheight);
%    \end{tikzpicture}}
%    }
%
%\makeatletter                   
%\def\printauthor{%                  
%    {\large \@author}}          
%\makeatother
%
%\author{%
%    Björn Schröder
%    }
%

%%%%%%%%%%%%%%%%%%%%%%%%%%%%%%%%%%%%%%%%%%%%%%%%%%%%%%%%%%%%%%%%%%%%%
%  Some chapterstyle alternatives for memoir class
%%%%%%%%%%%%%%%%%%%%%%%%%%%%%%%%%%%%%%%%%%%%%%%%%%%%%%%%%%%%%%%%%%%%%

%\chapterstyle{hangnum}
%\chapterstyle{companion}
%\chapterstyle{demo}
%\chapterstyle{brotherton}
%\chapterstyle{bianchi}
%\chapterstyle{dash}
%\chapterstyle{ell}
%\chapterstyle{demo3}
%
%\chapterstyle{pedersen}


%%%%%%%%%%%%%%%%%%%%%%%%%%%%%%%%%%%%%%%%%%%%%%%%%%%%%%%%%%%%%%%%%%%%%
%  Further customizing
%%%%%%%%%%%%%%%%%%%%%%%%%%%%%%%%%%%%%%%%%%%%%%%%%%%%%%%%%%%%%%%%%%%%%
\widowpenalty100000
\brokenpenalty100000

\definecolor{lightgray}{gray}{0.90}
\newcommand\cb[1]{\fcolorbox{lightgray}{lightgray}{#1}}

% nested items symbol
%\renewcommand{\labelitemi}{\tiny{\ding{110}}} 

\usepackage{enumitem}
\usepackage{pbox}

%%%%%%%%%%%%%%%%%%%%%%%%%%%%%%%%%%%%%%%%%%%%%%%%%%%%%%%%%%%%%%%%%%%%%
%  Modify the margin notes to flushleft
%%%%%%%%%%%%%%%%%%%%%%%%%%%%%%%%%%%%%%%%%%%%%%%%%%%%%%%%%%%%%%%%%%%%%
\marginparmargin{outer}
\let\oldmarginpar\marginpar
\renewcommand\marginpar[2][]{%
    \oldmarginpar{\mpjustification #2}}
%%%%%%%%%%%%%%%%%%%%%%%%%%%%%%%%%%%%%%%%%%%%%%%%%%%%%%%%%%%%%%%%%%%%%

\renewcommand{\rmdefault}{lsbj}

\newcommand{\pr}[1]{\textsuperscript{\textsc{\textcolor{ared}{#1}}}}
\newcommand{\pts}[1]{\textsuperscript{\textsc{\textcolor{blue}{#1}}}}

\DeclareRobustCommand{\dragon}{%
  \begingroup\normalfont
  \includegraphics[height=8pt]{dragonlogoviolett}%
  \endgroup
}

%\newcommand{\dragon}{%
%  \begingroup\normalfont
%  \includegraphics[height=\fontcharht\font`\B]{dragonlogoviolett}%
%  \endgroup
%}

\newcommand{\dragl}{\textsuperscript{\includegraphics[height=\fontcharht\font`\B]{dragonlogoviolett}}}

\begin{document} 

%%%%%%%%%%%%%%%%%%%%%%%%%%%%%%%%%%%%%%%%%%%%%%%%%%%%%%%%%%%%%%%%%%%%%
% Titlepages 
%%%%%%%%%%%%%%%%%%%%%%%%%%%%%%%%%%%%%%%%%%%%%%%%%%%%%%%%%%%%%%%%%%%%%
%\frontmatter
%\input{./child_cover.tex}
%\input{./child_titlepages.tex}
%\newpage
%\tableofcontents*
%\newpage
%\listoffigures
%\newpage
%\listoftables
%\newpage
%\mainmatter
%%%%%%%%%%%%%%%%%%%%%%%%%%%%%%%%%%%%%%%%%%%%%%%%%%%%%%%%%%%%%%%%%%%%%
% Put docuemtes here
%%%%%%%%%%%%%%%%%%%%%%%%%%%%%%%%%%%%%%%%%%%%%%%%%%%%%%%%%%%%%%%%%%%%%
%\input{./generate_tables.tex}
%\input{./nested_input.tex}


\clearpage
\thispagestyle{empty}

\section*{Units}


\begin{tabular}{lcccccccccl}
\toprule 
& \textsc{M}& \textsc{WS}& \textsc{BS}& \textsc{S}& \textsc{T}& \textsc{W}& \textsc{I}& \textsc{A}& \textsc{Ld}& \textsc{Troop Type}\\ \midrule
Morathi & 5 & 5 & 4 & 3 & 3 & 3 & 6 & 3 & 10 & Monstrous Cavalry\hyperref[rule:mostrouscavalry]{\pr{p83}}\\ 
Sulephet (Dark Pegasus) & 8 & 4 & 0 & 4 & 4 & 3 & 4 & 3 & 6 & \\
\midrule
Death Hag & 5 & 6 & 6 & 4 & 3 & 2 & 7 & 3 & 9 & Infantry\\
Cauldron of Blood & 5 & - & - & 5 & 6 & 3 & - & - & - & Chariot\pr{p86} (Armor Save $6+$)\\
Hag & - & 4 & 4 & 3 & - & - & 6 & 1 & - & \\
\midrule
Dark Rider & 5 & 4 & 4 & 3 & 3 & 1 & 5 & 1 & 8 & Cavalry\hyperref[rule:cavalry]{\pr{p82}}\\
Dark Steed & 9 & 3 & 0 & 3 & 3 & 1 & 4 & 1 & 5 & \\
\midrule
Witch Elves & 5 & 4 & 4 & 3 & 3 & 1 & 6 & 1 & 8 & Infantry\\
\midrule
Executioners & 5 & 5 & 4 & 4 & 3 & 1 & 5 & 1 & 9 & Infantry \\
\midrule
Shades & 5 & 5 & 5 & 3 & 3 & 1 & 5 & 1 & 8 & Infantry\\
\midrule
War Hydra & 6 & 4 & 4 & 5 & 5 & 5 & 2 & 3$+*$ & 6 & Monster\hyperref[rule:monster]{\pr{p85}} \\ 
\bottomrule
\end{tabular}


%----------------------------------------------------
\subsection*{\dragon Morathi\pr{p54}}
\marginpar{\color{blue}{375pts}}
Level 4 wizzard\\
Mount: Sulephet, the Dark Pegasus\\
\noindent
\begin{tabular}{lcccccccccl}
\toprule 
&
\textsc{M}&
\textsc{WS}&
\textsc{BS}&
\textsc{S}&
\textsc{T}&
\textsc{W}&
\textsc{I}&
\textsc{A}&
\textsc{Ld}&
\textsc{Troop Type}\\ \midrule
Morathi & 5 & 5 & 4 & 3 & 3 & 3 & 6 & 3 & 10 & Monstrous Cavalry\hyperref[rule:mostrouscavalry]{\pr{p83}}\\ Sulephet (Dark Pegasus) & 8 & 4 & 0 & 4 & 4 & 3 & 4 & 3 & 6 & \\
\bottomrule
\end{tabular}

\vspace{1em}
\begin{itemize}[noitemsep,nolistsep]
\item Always strikes first (Morathi only)
\item Enchanting Beauty
\item Fly\hyperref[rule:fly]{\dragon}
\item Hatred\hyperref[rule:hatred]{\dragon} (High Elves, Morathi only)
\item Hekarti's Blessing\hyperref[rule:hekbless]{\dragon}
\item Impale Attack\hyperref[rule:impaleattack]{\dragon} (Sulephet only)
\item Murderous Prowness\hyperref[rule:murderousprowness]{\dragon}
\end{itemize}

\begin{description}
\item[The First Sorceress\pr{DE p54}] +3 to all casting attempts
\item[Thousand and One Dark Blessings\pr{DE p54}] +4 Ward Save and Magic Resistance (2)
\item[\textsc{Heartrender and the Darksword}] 
Magic Weapon. Paired Weapons. Hits from this weapon have the Killing Blow\hyperref[rule:killingblow]{\dragon} special rule and are resolved at +2 Strength in the turn Morathi charges. In addition, a monster or character that suffers one or more unsaved Wounds from Heartrender and the Darksword reduces its Attacks, Strength and Toughness characteristics by one (to a minimum of 1) for each unsaved Wound. These penialties are applied at the end of the round of close combat in which the Wounds were sufferedm and last for the rmeainder of the game.
\item[Monstrous Cavalry] Cavalry\hyperref[rule:cavalry]{\pr{p82}} Rules apply but highest Wound characteristics are used. Stomp\hyperref[rule:stomp]{\pr{p76}} special rule. Mounsterous Ranks special rule.
\end{description}
%------------------------------------------------------
\subsection*{\dragon Death Hag\pr{DE p46} \marginpar{\color{blue}{275pts}}} 
\begin{tabular}{lcccccccccl}
\toprule  
&
\textsc{M}&
\textsc{WS}&
\textsc{BS}&
\textsc{S}&
\textsc{T}&
\textsc{W}&
\textsc{I}&
\textsc{A}&
\textsc{Ld}&
\textsc{Troop Type}\\ \midrule
Death Hag         & 5 & 6 & 6 & 4 & 3 & 2 & 7 & 3 & 9 & Infantry\\
Cauldron of Blood & 5 & - & - & 5 & 6 & 3 & - & - & - & Chariot\hyperref[rule:chariot]{\pr{p86}} (Armor Save 6+)\\
Hag               & - & 4 & 4 & 3 & - & - & 6 & 1 & - & \\
\bottomrule
\end{tabular}

\vspace{1em}
\begin{minipage}[t]{0.4\textwidth}
\subsubsection*{Death Hag\pts{85pts}}
\begin{itemize}[noitemsep,nolistsep]
\item Always strikes first\hyperref[rule:alwaysstrikesfirst]{\dragon}
\item Frenzy\hyperref[rule:frenzy]{\dragon}
\item Hatred\hyperref[rule:hatred]{\dragon} (High Elves)
\item Murderous Prowess\hyperref[rule:murderousprowness]{\dragon}
\item Poisoned Attacks\hyperref[rule:poisonedattacks]{\dragon}
\item Two Hand Weapons\hyperref[rule:twohandweapons]{\dragon}
\end{itemize}
\end{minipage}
\begin{minipage}[t]{0.6\textwidth}
\subsubsection*{Cauldron of Blood\pts{190pts}}
\begin{itemize}[noitemsep,nolistsep]
\item Always strikes first\hyperref[rule:alwaysstrikesfirst]{\dragon}
\item Frenzy\hyperref[rule:frenzy]{\dragon}
\item Hatred\hyperref[rule:hatred]{\dragon} (High Elves)
\item Murderous Prowess\hyperref[rule:murderousprowness]{\dragon}
\item Poisoned Attacks\hyperref[rule:poisonedattacks]{\dragon}
\item Chariot (Armor Save 6+)
\item Bloodshield of Khaine\hyperref[rule:bloodshield]{\dragon}
\item Fury of Khaine\hyperref[rule:furyofkhaine]{\dragon}
\item Large Target\hyperref[rule:largetarget]{\dragon}
\item Magic Resistance (1)\hyperref[rule:magicresistance]{\dragon}
\item Strength of Kaine\hyperref[rule:strengthofkhaine]{\dragon}
\item Terror\hyperref[rule:terror]{\dragon}
\item Will of the Gods\hyperref[rule:willofgods]{\dragon}
\end{itemize}
\end{minipage}

%------------------------------------------------------
\subsection*{\dragon Dark Riders\pr{DE p41}
\marginpar{\color{blue}{210pts}}}

\begin{tabular}{lcccccccccl}
\toprule  
&
\textsc{M}&
\textsc{WS}&
\textsc{BS}&
\textsc{S}&
\textsc{T}&
\textsc{W}&
\textsc{I}&
\textsc{A}&
\textsc{Ld}& 
\textsc{Troop Type}\\ \midrule
Dark Rider & 5 & 4 & 4 & 3 & 3 & 1 & 5 & 1 & 8 & Cavalry\hyperref[rule:]{\pr{p82}}\\
Dark Steed & 9 & 3 & 0 & 3 & 3 & 1 & 4 & 1 & 5 & \\
\bottomrule
\end{tabular}

\vspace{1em}

\begin{minipage}[t]{0.4\textwidth}
\begin{itemize}[noitemsep,nolistsep]
\item Always strikes first
\item Fast Cavalry
\item Hatred
\item Murderous Prowness
\item Repeater Crossbows\pr{DE p34}
\item Shields\hyperref[rule:]{\pr{p43}}
\end{itemize}
\end{minipage}
\begin{minipage}[b]{0.4\textwidth}
\resizebox{\textwidth}{!}{%
\begin{tabular}{ccc}
\toprule
Range & Strength & Special Rules \\
\midrule
24" & 3 & \pbox{20cm}{Armour Piercing\\Multiple Shots (2)} \\
\bottomrule
\end{tabular}
}
\end{minipage}

%------------------------------------------------------
\subsection*{\dragon Witch Elves\pr{DE p46}}
\marginpar{\small{Witch Elves under the Fury of Khaine spell influence have four attacks (Two Hand Weapons, 2 $\times$ Frenzy).  They re-roll all To-Hit rolls (Always stikes first) and all failed To-Wound rolls (Murderous Prowness). Each roll of a 6 is a automatic wound (Poisoned Attack).}}
\begin{tabular}{lcccccccccl}
\toprule  
&
\textsc{M}&
\textsc{WS}&
\textsc{BS}&
\textsc{S}&
\textsc{T}&
\textsc{W}&
\textsc{I}&
\textsc{A}&
\textsc{Ld}& \textsc{Troop Type}\\ \midrule
Witch Elves & 5 & 4 & 4 & 3 & 3 & 1 & 6 & 1 & 8 & Infantry\\
\bottomrule
\end{tabular}

\vspace{1em}

\begin{minipage}[t]{0.4\textwidth}
\begin{itemize}[noitemsep,nolistsep]
\item Standard Bearer\hyperref[rule:]{\pr{p94}}
\item Two Hand Weapons
\item Always strikes first
\item Frenzy
\end{itemize}
\end{minipage}
\begin{minipage}[t]{0.4\textwidth}
\begin{itemize}[noitemsep,nolistsep]
\item Hatred
\item Madness of Khaine
\item Murderous Prowness
\item Poisoned Attacks
\end{itemize}
\end{minipage}



\subsection*{\dragon Executioners of Har Ganeth\pr{DE p44}}
\marginpar{}
\begin{tabular}{lcccccccccl}
\toprule  
&
\textsc{M}&
\textsc{WS}&
\textsc{BS}&
\textsc{S}&
\textsc{T}&
\textsc{W}&
\textsc{I}&
\textsc{A}&
\textsc{Ld}& Troop Type\\ \midrule
Executioners & 5 & 5 & 4 & 4 & 3 & 1 & 5 & 1 & 9 & Infantry \\
\bottomrule
\end{tabular}

\vspace{1em}
\begin{itemize}[noitemsep,nolistsep]
\item Always strikes first
\item Killing Blow
\item Hatred
\item Murderous Prowness
\end{itemize}

%------------------------------------------------------
\subsection*{\dragon Shades\pr{DE p40}}
\marginpar{}
\begin{tabular}{lcccccccccl}
\toprule  
&
\textsc{M}&
\textsc{WS}&
\textsc{BS}&
\textsc{S}&
\textsc{T}&
\textsc{W}&
\textsc{I}&
\textsc{A}&
\textsc{Ld}& Troop Type\\ \midrule
Shades & 5 & 5 & 5 & 3 & 3 & 1 & 5 & 1 & 8 & Infantry\\
\bottomrule
\end{tabular}

\vspace{1em}
\begin{minipage}[t]{0.4\textwidth}
\begin{itemize}[noitemsep,nolistsep]
\item Hand Weapons
\item Repeater Crossbow
\item Hatred
\item Scouts
\item Murderous Prowness
\item Skirmishers
\end{itemize}
\end{minipage}
\begin{minipage}[b]{0.4\textwidth}
\resizebox{\textwidth}{!}{%
\begin{tabular}{ccc}
\toprule
Range & Strength & Special Rules \\
\midrule
24" & 3 & \pbox{20cm}{Armour Piercing\\Multiple Shots (2)} \\
\bottomrule
\end{tabular}
}
\end{minipage}



\subsection*{\dragon War Hydra\pr{DE p49}}
\marginpar{}
\begin{tabular}{lcccccccccl}
\toprule  
&
\textsc{M}&
\textsc{WS}&
\textsc{BS}&
\textsc{S}&
\textsc{T}&
\textsc{W}&
\textsc{I}&
\textsc{A}&
\textsc{Ld}&
\textsc{Troop Type}\\ \midrule
War Hydra & 6 & 4 & 4 & 5 & 5 & 5 & 2 & 3+* & 6 & Monster\hyperref[rule:monster]{\pr{p85}} \\ 
\bottomrule
\end{tabular}

\marginpar{\small{Monsters have a more destructive version of Stomp, called Thunderstomp.\hyperref[rule:thunderstomp]{\pr{p76}}}}
\vspace{1em}
\begin{itemize}[noitemsep,nolistsep]
\item If One Head is Severed \ldots
\item \ldots Another Takes its Place
\item Large Target
\item Scaly Skin (4+)
\item Terror
\item Fiery Breath\textsuperscript{\color{blue}{20pts~}}\hyperref[rule:]{\pr{p67}}
\end{itemize}

\section*{Special Rules}
\marginpar{}
\begin{description}
\item[\textsc{Murderous Prowness\pr{DE p34}}]\phantomsection\label{rule:murderousprowness} Models with this special rule (but not their mounts) re-roll all To Wound rolls of a 1 when making close combat attacks.
\item[\textsc{Enchanting Beauty\pr{DE p54}}]\phantomsection\label{rule:enchantingbeauty} Model in base contact must pass Leadership or be reduced to WS 1 till end of phase
\item[\textsc{Hekarti's Blessing\pr{DE p34}}]\phantomsection\label{rule:hekbless} Models with tis special rule add +1 to all at tempts to cast spells from the Lore of Dark Magic.  
\item[\textsc{Eternal Hatred\pr{DE p34}}]\phantomsection\label{rule:eternalhatred} A model with this speical rule has the Hatred\hyperref[rule:hatred]{\dragon} special rule. In addition, its Hatred applies in every round of close combat, not just the first.
\item[\textsc{Impale Attack\pr{HE p50}}]\phantomsection\label{rule:impaleattack} On a turn in which it charges, a Dark Pegasus; close combat attacks are resolved at +1 Strength. 
\item[\textsc{Will of the Gods\pr{DE p47}}]\phantomsection\label{rule:willofgods} The model has no steeds but uses its own Movement value and can march, join units as if it were a character (but must be placed in the centre front rank). Only one model with this special rule can join each unit.
\item[\textsc{Bloodshield of Khaine\pr{DE p47}}]\phantomsection\label{rule:bloodshield} 4+ Ward save. Witch Elves, Hags, Death Hags (including Hellebron) in the same unit or mounted on it have a 5+ ward save, and all other models in the unit have a 6+ ward save.
\item[\textsc{Strength of Khaine\pr{DE p47}}]\phantomsection\label{rule:strengthofkhaine} Friendly models with the Murderous Prowess\hyperref[rule:murderousprowness]{\dragon} special rule in units within 6" re-roll all failed To Wound rolls. 
\item[\textsc{Fury of Khaine\pr{DE p47}}]\phantomsection\label{rule:furyofkhaine} Innate bound augment spell (level 3) that targets a single unit within 12" The target gains the Frenzy\hyperref[rule:frenzy]{\dragon} special rule until the start of the Cauldron of Blood's next Magic phase. If the target already has the Frenzy special rule, that Frenzy grants +2 Attacks to every model in the unit instead of just +1. Not cumulative with Witchbrew.
\item[\textsc{Madness of Kaine\pr{DE p46}}]\phantomsection\label{rule:madnessofkhaine} At the end of each of your turns, roll a D6 for each of your characters that is in a unit of Witch Elves (do not roll for Khainite Assassins, Shadowblade, Death Hags or Hellebron - they'e learnt how to survive in such company). On a score of 4+, nothing happens. On a score of 3 or less, that characters immediately suffers D6 Strength 3 hits as the Witch Elves lose all control and turn on their ally. 

\item[\textsc{Always Strikes First\pr{p66}}]\phantomsection\label{rule:alwaysstrikesfirst} Unit always strikes first. If the unit's Initiative is equal to or higher than the enemy's, he can re-roll failed misses when striking in close combat.
\item[\textsc{Cavalry\pr{p82}}]\phantomsection\label{rule:cavalry} The rider and mount use their own Weapon Skill, Strength, Initiative and Attacks characteristics when they attack. Each can attack any opponent that the cavalry model is in base contact with. The mount's Wounds and Toughness are never used. Riders Weapon Skill is used. Cavalry have the Swiftsride rule.\hyperref[rule:swiftstride]{\pr{p76}} Unless otherwise noted, special rules that apply to the mount do not normally (see exceptions in rule book) also apply to the rider. \\ Rider's armour saves are used. A cavalry models armour save is treated as being one point better. If the mount has barding, the riders's armour save is increased by two points.
\item[\textsc{Fast Cavalry\pr{p68}}]\phantomsection\label{rule:fastcavalry} Vanguard deployment rule.\pr{p79} Free Reform. Feigned Flight. Fire on the March. 
\item[\textsc{Monstrous Cavalry\pr{p83}}]\phantomsection\label{rule:mostrouscavalry} All the cavalry rules apply to monstrous cavalry rules with one exception - monstrous cavalry always use the highest Wounds characteristic the model has rather than automatically using the rider's - indecd this will normally mean that the model uses the mount's Wounds characteristic.
\item[\textsc{Fear\pr{p69}}]\phantomsection\label{rule:fear} At the start of each Close Combat round, a unit that is in base contact with one or more enemy models that cause Fear must take a Leadership test, before any blows are struck. 
\item[\textsc{Fly\pr{p70}}]\phantomsection\label{rule:fly} All flyers have the Swiftstride\hyperref[rule:swiftstride]{\dragon} special rule. Movement begins and ends on the ground and moves up to 10". Only entire units that can both walk and fly can do both. A flying charge is calculated with a movement of 10". Units can perform a Flying March of up to 20". Flyers always move on the ground when they flee or pursue, but still benefit from Swiftstride. Flying Cavalry are treated as Fast Cavalry with the Fly special rule.
\item[\textsc{Frenzy\pr{p70}}]\phantomsection\label{rule:frenzy} To represent their fighting fury and lack of self-preservation instincts, Frenzied troops have the Extra Attack and Immune to Psychology\hyperref[rule:immune]{\dragon} special rules.
\item[\textsc{Hatred\pr{p71}}]\phantomsection\label{rule:hatred} Re-roll Missed attacks during the first round of close combat against High Elves. 
\item[\textsc{Immune to Psychology\pr{p69}}]\phantomsection\label{rule:immune} Automatically passes all Panic, Fear and Terror tests.
\item[\textsc{Impact Hits\pr{p71}}]\phantomsection\label{rule:impacthits} D6 Hits on charge resolved first in close combat at the strength of the model, with hits distributed as if they were shooting attacks, and any unsaved wounds do count towards combat results. 
\item[\textsc{Killing Blow\pr{p72}}]\phantomsection\label{rule:killingblow} Rolls of 6 to wound in close combat automatically slays the opponent regardless of the number of wounds, no armour saves allowed. Only effective against infantry, cavalry and war beasts.
\item[\textsc{Large Target\pr{p72}}]\phantomsection\label{rule:largetarget} Cannot claim cover modifiers for obstacles. If General or Battle Standard Bearer is a Large Target (or is mounted on one), then the range of their respective Inspiring Presence and Hold Your Ground! abilities is increased from 12"to 18".
\item[\textsc{Magic Resistance\pr{p72} (1)}]\phantomsection\label{rule:magicresistance} Bonus to ward save against spells. E.g. a 5+ becomes a 3+.
\item[\textsc{Monstrous\pr{p81}}]\phantomsection\label{rule:monstrous} Have the Stomp and Swiftstride special rule. Needs only 3 units to form ranks and 6 to form a horde.
\item[\textsc{Poisoned Attacks\pr{p73}}]\phantomsection\label{rule:poisonedattacks} Attacks wound automatically on a to-hit roll of 6.
\item[\textsc{Scaly Skin\pr{p75} 3+}]\phantomsection\label{rule:scalyskin} Armour save of 3+ 
\item[\textsc{Scouts\pr{p79}}]\phantomsection\label{rule:scouts} Scouts are set up after all other non-Scout units from both armies have been deployed. they can be set up either in their controlling player' deployment zone, or anywhere on the battlefield more than 12" away from the enemy. If deployed in this second way, Scouts cannot declare a charge in the first turn if their side goes first.
\item[\textsc{Skirmishers\pr{p77}}]\phantomsection\label{rule:skirmish} Skirmish Formation, Skirmishers and Charging, Free Reform, Fire on the March, Light Troops 
\item[\textsc{Stomp\pr{p76}}]\phantomsection\label{rule:stomp} Can make a Stomp in addition to other close combat attacks with the Always strikes last rule. It inflicts 1 automatic hit at the models strength.
\item[\textsc{Thunderstomp\pr{p76}}]\phantomsection\label{rule:thunderstomp} A Thunderstomp makes D6 hits on the target unit.
\item[\textsc{Swiftstride\pr{p76}}]\phantomsection\label{rule:swiftstride} When charging uses 3D6 and discard the lowest result and add to their move value. When fleeing and pursuing, use 3D6 and discard the lowest result.
\item[\textsc{Terror\pr{p78}}]\phantomsection\label{rule:terror} Models that cause Terror also cause Fear. Terror-causing models are themselves immune to both Fear and Terror. If a Terror causing creature declares a charge the target unit must immediately take a panic test. 
%\item[\textsc{+++}] 
\end{description}

%\input{./child_watermark.tex}
%\input{./child_examples.tex}
%\input{./child_tables.tex}
%\input{./child_pictures.tex}
%\input{./child_appendix.tex}

%\blindmathtrue
%blinddocument

\end{document}
